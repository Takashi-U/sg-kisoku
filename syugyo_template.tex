\documentclass{jsarticle}
\usepackage{kisoku_base_macros}

\begin{document}
\title{就業規則}
\author{Takashi Uchibe}
\maketitle

\section{総則}
\label{就規_章_総則}

\subsection{目的}
\label{就規_条_目的}
\subsubsection{}
\label{就規_項_目的}
この就業規則(以下「規則」という。)は、労働基準法(以下「労基法」という。)第89条に基づき、●●株式会社の従業員の就業に関する事項を定めるものである
\subsubsection{}
\label{就規_項_目的その他}
この規則に定めた事項のほか、就業に関する事項については、労基法その他の法令の定めによる。

\subsection{適用範囲}
\label{就規_条_適用範囲}
\subsubsection{}
\label{就規_項_適用範囲}
この規則は、●●株式会社の従業員に適用する。
\subsubsection{}
\label{就規_項_適用範囲(パート)}
パートタイム従業員の就業に関する事項については、別に定めるところによる。
\subsubsection{}
\label{就規_項_適用範囲その他}
\ref{就規_項_適用範囲(パート)}については、別に定める規則に定めのない事項は、この規則を適用する。

\subsection{規則の遵守}
\label{就規_条_規則の遵守}
\subsubsection{}
\label{就規_項_規則の遵守}
会社は、この規則に定める労働条件により、従業員に就業させる義務を負う。また、従業員は、この規則を遵守しなければならない。

\section{採用、異動等}
\label{就規_章_採用、異動等}

\subsection{採用手続}
\label{就規_条_採用手続}
\subsubsection{}
\label{就規_項_採用手続}
会社は、入社を希望する者の中から選考試験を行い、これに合格した者を採用する。

\subsection{採用時の提出書類}
\label{就規_条_採用時の提出書類}
\subsubsection{}
\label{就規_項_採用時の提出書類}
従業員として採用された者は、採用された日から◆◆週間以内に次の書類を提出しなければならない。
\begin{enumerate}
  \item 履歴書
  \item 住民票記載事項証明書
  \item 自動車運転免許証の写し(ただし、自動車運転免許証を有する場合に限る。)
  \item 資格証明書の写し(ただし、何らかの資格証明書を有する場合に限る。)
  \item その他会社が指定するもの
\end{enumerate}
\subsubsection{}
\label{就規_項_変更事項の届出}
前項の定めにより提出した書類の記載事項に変更を生じたときは、速やかに書面で会社に変更事項を届け出なければならない。

\subsection{試用期間}
\label{就規_条_試用期間}
\subsubsection{}
\label{就規_項_試用期間}
従業員として新たに採用した者については、採用した日から◆◆か月間を試用期間とする。
\subsubsection{}
\label{就規_項_試用期間の短縮等}
前項について、会社が特に認めたときは、この期間を短縮し、又は設けないことがある。
\subsubsection{}
\label{就規_項_試用期間中の解雇}
試用期間中に従業員として不適格と認めた者は、解雇することがある。ただし、入社後14日を経過した者については、\ref{就規_項_解雇の予告}によって行う。
\subsubsection{}
\label{就規_項_試用期間の勤続年数への通算}
試用期間は、勤続年数に通算する。

\subsection{労働条件の明示}
\label{就規_条_労働条件の明示}
\subsubsection{}
\label{就規_項_労働条件の明示}
会社は、従業員を採用するとき、採用時の賃金、就業場所、従事する業務、労働時間、休日、その他の労働条件を記した労働条件通知書及びこの規則を交付して労働条件を明示するものとする。

\subsection{人事異動}
\label{就規_条_人事異動}
\subsubsection{}
\label{就規_項_人事異動}
会社は、業務上必要がある場合に、従業員に対して就業する場所及び従事する業務の変更を命ずることがある。
\subsubsection{}
\label{就規_項_在籍出向}
会社は、業務上必要がある場合に、従業員を在籍のまま関係会社へ出向させることがある。
\subsubsection{}
\label{就規_項_人事異動・在籍出向の正当な理由無き拒否の禁止}
\ref{就規_項_人事異動}および\ref{就規_項_在籍出向}の場合、従業員は正当な理由なくこれを拒むことはできない。

\subsection{休職}
\label{就規_条_休職}
\subsubsection{}
\label{就規_項_休職}
従業員が、次のいずれかに該当するときは、所定の期間休職とする。
\begin{enumerate}
  \item 業務外の傷病による欠勤が◆◆か月を超え、なお療養を継続する必要があるため勤務できないとき ◆◆年以内\label{enum:就規_号_業務外傷病による休職}
  \item 前号のほか、特別な事情があり休職させることが適当と認められるとき 必要な期間
\end{enumerate}
\subsubsection{}
\label{就規_項_休職からの復帰}
休職期間中に休職事由が消滅したときは、原則として元の職務に復帰させる。ただし、元の職務に復帰させることが困難又は不適当な場合には、他の職務に就かせることがある。
\subsubsection{}
\label{就規_項_休職期間満了による退職}
\ref{就規_項_休職}\ref{enum:就規_号_業務外傷病による休職}により休職し、休職期間が満了してもなお傷病が治癒せず就業が困難な場合は、休職期間の満了をもって退職とする。

\section{服務規律}
\label{就規_章_服務規律}

\subsection{服務}
\label{就規_条_服務}
\subsubsection{}
\label{就規_項_服務}
従業員は、職務上の責任を自覚し、誠実に職務を遂行するとともに、会社の指示命令に従い、職務能率の向上及び職場秩序の維持に努めなければならない。

\subsection{遵守事項}
\label{就規_条_遵守事項}
\subsubsection{}
\label{就規_項_遵守事項}
従業員は、以下の事項を守らなければならない。
\begin{enumerate}
  \item 許可なく職務以外の目的で会社の施設、物品等を使用しないこと。
  \item 職務に関連して自己の利益を図り、又は他より不当に金品を借用し、若しくは贈与を受ける等不正な行為を行わないこと。
  \item 勤務中は職務に専念し、正当な理由なく勤務場所を離れないこと。
  \item 会社の名誉や信用を損なう行為をしないこと。
  \item 在職中及び退職後においても、業務上知り得た会社、取引先等の機密を漏洩しないこと。
  \item 許可なく他の会社等の業務に従事しないこと。
  \item 酒気を帯びて就業しないこと。
  \item その他従業員としてふさわしくない行為をしないこと。
\end{enumerate}

\subsection{セクシュアルハラスメントの禁止}
\label{就規_条_セクシュアルハラスメントの禁止}
\subsubsection{}
\label{就規_項_セクシュアルハラスメントの禁止}
性的言動により、他の従業員に不利益や不快感を与えたり、就業環境を害するようなことをしてはならない。

\subsection{個人情報保護}
\label{就規_条_個人情報保護}
\subsubsection{}
\label{就規_項_個人情報保護}
従業員は、会社及び取引先等に関する情報の管理に十分注意を払うとともに、自らの業務に関係のない情報を不当に取得してはならない。
\subsubsection{}
\label{就規_項_退職時個人情報取扱}
従業員は、職場又は職種を異動あるいは退職するに際して、自らが管理していた会社及び取引先等に関するデータ・情報書類等を速やかに返却しなければならない。

\subsection{始業及び終業時刻の記録}
\label{就規_条_始業及び終業時刻の記録}
\subsubsection{}
\label{就規_項_始業及び終業時刻の記録}
従業員は、始業及び終業時にタイムカードを自ら打刻し、始業及び終業の時刻を記録しなければならない。

\subsection{遅刻、早退、欠勤等}
\label{就規_条_遅刻、早退、欠勤等}
\subsubsection{}
\label{就規_項_遅刻、早退、欠勤等}
従業員は遅刻、早退若しくは欠勤をし、又は勤務時間中に私用で事業場から外出する際は、事前に◆◆に対し申し出るとともに、承認を受けなければならない。ただし、やむを得ない理由で事前に申し出ることができなかった場合は、事後に速やかに届出をし、承認を得なければならない。
\subsubsection{}
\label{就規_項_遅刻、早退、欠勤等による賃金控除}
前項の場合は、\ref{就規_条_賃金}の賃金規程に定めるところにより、原則として不就労分に対応する賃金は控除する。
\subsubsection{}
\label{就規_項_診断書の提出}
傷病のため継続して◆◆日以上欠勤するときは、医師の診断書を提出しなければならない。

\section{労働時間、休憩及び休日}
\label{就規_章_労働時間、休憩及び休日}

\subsection{労働時間及び休憩時間}
\label{就規_条_労働時間及び休憩時間}
\subsubsection{}
\label{就規_項_労働時間及び休憩時間}
労働時間は、1週間については40時間、1日については8時間とする。
\subsubsection{}
\label{就規_項_始業・終業の時刻及び休憩時間}
始業・終業の時刻及び休憩時間は、次のとおりとする。ただし、業務の都合その他やむを得ない事情により、これらを繰り上げ、又は繰り下げることがある。この場合、◆◆が前日までに従業員に通知する。
  \begin{table}[htb]
    \centering
    \begin{tabular}{|c|c|c|c|} \hline
      勤務形態 & 始業 & 終業 & 休憩時間 \\ \hline \hline
      一般勤務 & ◆時◆分 & ◆時◆分 & ◆時◆分から◆時◆分まで \\ \hline
               & ◆時◆分 & ◆時◆分 & ◆時◆分から◆時◆分まで \\ \cline{2-4}
      交代勤務 & ◆時◆分 & ◆時◆分 & ◆時◆分から◆時◆分まで \\ \cline{2-4}
               & ◆時◆分 & ◆時◆分 & ◆時◆分から◆時◆分まで \\ \hline
    \end{tabular}
  \end{table}
\subsubsection{}
\label{就規_項_交替勤務のシフト通知}
交替勤務における各従業員の勤務は、別に定めるシフト表により、前月の   日までに各従業員に通知する。
\subsubsection{}
\label{就規_項_交替勤務における就業番}
交替勤務における就業番は原則として◆日ごとに◆番を◆番に、◆番を◆番に、◆番を◆番に転換する。
\subsubsection{}
\label{就規_項_一般勤務と交替勤務との変更}
一般勤務から交替勤務へ、交替勤務から一般勤務への勤務形態の変更は、原則として休日又は非番明けに行うものとし、前月の◆日前までに◆が従業員に通知する。

\subsection{休日}
\label{就規_条_休日}
\subsubsection{}
\label{就規_項_休日}
休日は、次のとおりとする。
\begin{enumerate}
  \item 土曜日及び日曜日
  \item 国民の祝日(日曜日と重なったときは翌日)
  \item 年末年始(12月◆日~1月◆日)
  \item 夏季休日(◆月◆日~◆月◆日)
  \item その他会社が指定する日
\end{enumerate}
\subsubsection{}
\label{就規_項_休日の振替}
業務の都合により会社が必要と認める場合は、あらかじめ前項の休日を他の日と振り替えることがある。

\subsection{時間外及び休日労働等}
\label{就規_条_時間外及び休日労働等}
\subsubsection{}
\label{就規_項_時間外及び休日労働等}
業務の都合により、\ref{就規_条_労働時間及び休憩時間}の所定労働時間を超え、又は\ref{就規_条_休日}の所定休日に労働させることがある。
\subsubsection{}
\label{就規_項_36協定}
前項の場合、法定労働時間を超える労働又は法定休日における労働については、あらかじめ会社は従業員の過半数代表者と書面による労使協定を締結するとともに、これを所轄の労働基準監督署長に届け出るものとする。
\subsubsection{}
\label{就規_項_女性・年少者の時間外・休日労働}
妊娠中の女性、産後1年を経過しない女性従業員であって請求した者及び18歳未満の者については、\ref{就規_項_36協定}による時間外労働又は休日若しくは深夜(午後10時から午前5時まで)労働に従事させない。
\subsubsection{}
\label{就規_項_天災地変等}
災害その他避けることのできない事由によって臨時の必要がある場合には、\ref{就規_項_時間外及び休日労働等}から\ref{就規_項_女性・年少者の時間外・休日労働}までの制限を超えて、所定労働時間外又は休日に労働させることがある。

\section{休暇等}
\label{就規_章_休暇等}

\subsection{年次有給休暇}
\label{就規_条_年次有給休暇}
\subsubsection{}
\label{就規_項_年次有給休暇}
採用日から6か月間継続勤務し、所定労働日の8割以上出勤した従業員に対しては、10日の年次有給休暇を与える。その後1年間継続勤務するごとに、当該1年間において所定労働日の8割以上出勤した従業員に対しては、下の表のとおり勤続期間に応じた日数の年次有給休暇を与える。
  \begin{table}[htb]
    \centering
    \begin{tabular}{|c|c|c|c|c|c|c|c|} \hline
      & & 1年 & 2年 & 3年 & 4年 & 5年 & 6年 \\
      勤続期間 &  6か月 & 6か月 & 6か月 & 6か月 & 6か月 & 6か月 & 6か月以上 \\ \hline \hline
      付与日数 & 10日 & 11日 & 12日 & 14日 & 16日 & 18日 & 20日 \\ \hline
    \end{tabular}
  \end{table}
\subsubsection{}
\label{就規_項_年次有給休暇の比例付与}
\ref{就規_項_年次有給休暇}の規定にかかわらず、週所定労働時間30時間未満であり、かつ、週所定労働日数が4日以下(週以外の期間によって所定労働日数を定める従業員については年間所定労働日数が216日以下)の従業員に対しては、下の表のとおり所定労働日数及び勤続期間に応じた日数の年次有給休暇を与える。
  \begin{table}[htb]
    \centering
    \begin{tabular}{|c|c|c|c|c|c|c|c|c|} \hline
      \multicolumn{2}{|l|}{} & \multicolumn{7}{|c|}{勤続年数} \\ \hline
      週所定 & 1年間の所定 &  & 1年 & 2年 & 3年 & 4年 & 5年 & 6年 \\ 
      労働日数 & 労働日数 &  6か月 & 6か月 & 6か月 & 6か月 & 6か月 & 6か月 & 6か月以上 \\ \hline \hline
      4日 & 169日~216日 & 7日 & 8日 & 9日 & 10日 & 12日 & 13日 & 15日 \\ \hline
      3日 & 121日~168日 & 5日 & 6日 & 6日 & 8日 & 9日 & 10日 & 11日 \\ \hline
      2日 & 73日~120日 & 3日 & 4日 & 4日 & 5日 & 6日 & 6日 & 7日 \\ \hline
      1日 & 48日~72日 & 1日 & 2日 & 2日 & 2日 & 3日 & 3日 & 3日 \\ \hline
    \end{tabular}
  \end{table}
\subsubsection{}
\label{就規_項_年次有給休暇の時季指定}
\ref{就規_項_年次有給休暇}又は\ref{就規_項_年次有給休暇の比例付与}の年次有給休暇は、従業員があらかじめ請求する時季に取得させる。ただし、従業員が請求した時季に年次有給休暇を取得させることが事業の正常な運営を妨げる場合は、他の時季に取得させることがある。
\subsubsection{}
\label{就規_項_年次有給休暇の計画的付与}
\ref{就規_項_年次有給休暇の時季指定}の規定にかかわらず、従業員代表との書面による協定により、各従業員の有する年次有給休暇日数のうち5日を超える部分について、あらかじめ時季を指定して取得させることがある。
\subsubsection{}
\label{就規_項_年次有給休暇の出勤率算定}
\ref{就規_項_年次有給休暇}及び\ref{就規_項_年次有給休暇の比例付与}の出勤率の算定に当たっては、下記の期間については出勤したものとして取り扱う。
  \begin{enumerate}
    \item 年次有給休暇を取得した期間
    \item 産前産後の休業期間
    \item 育児休業、介護休業等育児又は家族介護を行う労働者の福祉に関する法律(平成3年法律第76号。以下「育児・介護休業法」という。)に基づく育児休業及び介護休業した期間
    \item 業務上の負傷又は疾病により療養のために休業した期間
  \end{enumerate}
\subsubsection{}
\label{就規_項_年次有給休暇の繰越}
付与日から1年以内に取得しなかった年次有給休暇は、付与日から2年以内に限り繰り越して取得することができる。
\subsubsection{}
\label{就規_項_繰越分年次有給休暇の消費}
\ref{就規_項_年次有給休暇の繰越}について、繰り越された年次有給休暇とその後付与された年次有給休暇のいずれも取得できる場合には、繰り越された年次有給休暇から取得させる。
\subsubsection{}
\label{就規_項_年次有給休暇の残数通知}
会社は、毎月の賃金計算締切日における年次有給休暇の残日数を、当該賃金の支払明細書に記載して各従業員に通知する。

\subsection{年次有給休暇の時間単位での付与}
\label{就規_条_年次有給休暇の時間単位での付与}
\subsubsection{}
\label{就規_項_年次有給休暇の時間単位での付与}
従業員代表との書面による協定に基づき、\ref{就規_条_年次有給休暇}の年次有給休暇の日数のうち、1年について5日の範囲で次により時間単位の年次有給休暇(以下「時間単位年休」という。)を付与する。
  \begin{enumerate}
    \item 時間単位年休付与の対象者は、すべての従業員とする。
    \item 時間単位年休を取得する場合の、1日の年次有給休暇に相当する時間数は、以下のとおりとする。
    \begin{enumerate}
      \item 所定労働時間が5時間を超え6時間以下の者…6時間
      \item 所定労働時間が6時間を超え7時間以下の者…7時間
      \item 所定労働時間が7時間を超え8時間以下の者…8時間
    \end{enumerate}
    \item 時間単位年休は1時間単位で付与する。
    \item 本条の時間単位年休に支払われる賃金額は、所定労働時間労働した場合に支払われる通常の賃金の1時間当たりの額に、取得した時間単位年休の時間数を乗じた額とする。
    \item 上記以外の事項については、\ref{就規_条_年次有給休暇}の年次有給休暇と同様とする。
  \end{enumerate}

\subsection{産前産後の休業}
\label{就規_条_産前産後の休業}
\subsubsection{}
\label{就規_項_産前の休業}
6週間(多胎妊娠の場合は14週間)以内に出産予定の女性従業員から請求があったときは、休業させる。
\subsubsection{}
\label{就規_項_産後の休業}
産後8週間を経過していない女性従業員は、就業させない。
\subsubsection{}
\label{就規_項_産後6週間経過後の復帰}
\ref{就規_項_産後の休業}の規定にかかわらず、産後6週間を経過した女性従業員から請求があった場合は、その者について医師が支障がないと認めた業務に就かせることがある。

\subsection{母性健康管理の措置}
\label{就規_条_母性健康管理の措置}
\subsubsection{}
\label{就規_項_母性健康管理のための受診等}
妊娠中又は出産後1年を経過しない女性従業員から、所定労働時間内に、母子保健法(昭和40年法律第141号)に基づく保健指導又は健康診査を受けるために申出があったときは、次の範囲で時間内通院を認める。
  \begin{enumerate}
    \item 産前の場合
      \begin{enumerate}
        \item 妊娠23週まで…4週に1回
        \item 妊娠24週から35週まで…2週に1回
        \item 妊娠36週から出産まで…1週に1回
      \end{enumerate}
    ただし、医師又は助産師(以下「医師等」という。)がこれと異なる指示をしたときには、その指示により必要な時間
  \end{enumerate}
  \begin{enumerate}
    \item 産後(1年以内)の場合 \\
    医師等の指示により必要な時間
  \end{enumerate}
\subsubsection{}
\label{就規_項_医師等の指導に基づく母性健康管理の措置}
妊娠中又は出産後1年を経過しない女性従業員から、保健指導又は健康診査に基づき勤務時間等について医師等の指導を受けた旨申出があった場合、次の措置を講ずる。
  \begin{enumerate}
    \item 妊娠中の通勤緩和措置として、通勤時の混雑を避けるよう指導された場合は、原則として1時間の勤務時間の短縮又は1時間以内の時差出勤を認める。
    \item 妊娠中の休憩時間について指導された場合は、適宜休憩時間の延長や休憩の回数を増やす。
    \item 妊娠中又は出産後の女性従業員が、その症状等に関して指導された場合は、医師等の指導事項を遵守するための作業の軽減や勤務時間の短縮、休業等の措置をとる。
  \end{enumerate}

\subsection{}
\label{就規_条_育児時間及び生理休暇}
\subsubsection{}
\label{就規_項_育児時間}
1歳に満たない子を養育する女性従業員から請求があったときは、休憩時間のほか1日について2回、1回について30分の育児時間を与える。
\subsubsection{}
\label{就規_項_生理休暇}
生理日の就業が著しく困難な女性従業員から請求があったときは、必要な期間休暇を与える。

\subsection{育児・介護休業、子の看護休暇等}
\label{就規_条_育児・介護休業、子の看護休暇等}
\subsubsection{}
\label{就規_項_育児・介護休業、子の看護休暇等}
従業員のうち必要のある者は、育児・介護休業法に基づく育児休業、介護休業、子の看護休暇、介護休暇、育児のための所定外労働の免除、育児・介護のための時間外労働及び深夜業の制限並びに所定労働時間の短縮措置等(以下「育児・介護休業等」という。)の適用を受けることができる。
\subsubsection{}
\label{就規_項_育児・介護休業、子の看護休暇等の取扱}
育児休業、介護休業等の取扱いについては、「育児・介護休業等に関する規則」で定める。

\subsection{慶弔休暇}
\label{就規_条_慶弔休暇}
\subsubsection{}
\label{就規_項_慶弔休暇}
従業員が申請した場合は、次のとおり慶弔休暇を与える。
  \begin{table}[htb]
    \centering
    \begin{tabular}{|l|r|} \hline
      本人が結婚したとき & ◆日 \\ \hline
      妻が出産したとき & ◆日 \\ \hline
      配偶者、子又は父母が死亡したとき & ◆日 \\ \hline
      兄弟姉妹、祖父母、配偶者の父母又は兄弟姉妹が死亡したとき & ◆日 \\ \hline
    \end{tabular}
  \end{table}

\subsection{裁判員等のための休暇}
\label{就規_条_裁判員等のための休暇}
\subsubsection{}
\label{就規_項_裁判員等のための休暇}
従業員が裁判員若しくは補充裁判員となった場合又は裁判員候補者となった場合には、次のとおり休暇を与える。
  \begin{table}[htb]
    \centering
    \begin{tabular}{|l|c|} \hline
      裁判員又は補充裁判員となった場合 & 必要な日数 \\ \hline
      裁判員候補者となった場合 & 必要な時間 \\ \hline
    \end{tabular}
  \end{table}

\section{賃金}
\label{就規_章_賃金}

\subsection{賃金}
\label{就規_条_賃金}
\subsubsection{}
\label{就規_項_賃金}
賃金は別に定める賃金規程の定めるところによる。

\section{定年、退職及び解雇}
\label{就規_章_定年、退職及び解雇}

\subsection{定年等}
\label{就規_条_定年等}
\subsubsection{}
\label{就規_項_定年}
従業員の定年は、満60歳とし、定年に達した日の属する月の末日をもって退職とする。
\subsubsection{}
\label{就規_項_定年再雇用}
定年後も引き続き雇用されることを希望する従業員については満65歳までの間で本人の希望する期間、継続雇用する。
\label{就規_項_定年再雇用の契約}
継続雇用の雇用条件については、本人と会社の協議により別途契約により定める者とする。

\subsection{退職}
\label{就規_条_退職}
\subsubsection{}
\label{就規_項_退職}
\ref{就規_条_定年等}に定めるもののほか、従業員が次のいずれかに該当するときは、退職とする。
  \begin{enumerate}
    \item 退職を願い出て会社が承認したとき、又は退職願を提出して14日を経過したとき
    \item 期間を定めて雇用されている場合、その期間を満了したとき
    \item \ref{就規_条_休職}に定める休職期間が満了し、なお休職事由が消滅しないとき
    \item 死亡したとき
  \end{enumerate}
\subsubsection{}
\label{就規_項_退職事由の証明}
従業員が退職し、又は解雇された場合、その請求に基づき、使用期間、業務の種類、地位、賃金又は退職の事由を記載した証明書を遅滞なく交付する。

\subsection{解雇}
\label{就規_条_解雇}
\subsubsection{}
\label{就規_項_解雇}
従業員が次のいずれかに該当するときは、解雇することがある。
  \begin{enumerate}
    \item 勤務状況が著しく不良で、改善の見込みがなく、従業員としての職責を果たし得ないとき。
    \item 勤務成績又は業務能率が著しく不良で、向上の見込みがなく、他の職務にも転換できない等就業に適さないとき。
    \item 業務上の負傷又は疾病による療養の開始後3年を経過しても当該負傷又は疾病が治らない場合であって、従業員が傷病補償年金を受けているとき又は受けることとなったとき(会社が打ち切り補償を支払ったときを含む。)。
    \item 精神又は身体の障害により業務に耐えられないとき。
    \item 試用期間における作業能率又は勤務態度が著しく不良で、従業員として不適格であると認められたとき。
    \item \ref{就規_項_懲戒解雇}に定める懲戒解雇事由に該当する事実が認められたとき。
    \item 事業の運営上又は天災事変その他これに準ずるやむを得ない事由により、事業の縮小又は部門の閉鎖等を行う必要が生じ、かつ他の職務への転換が困難なとき。
    \item その他前各号に準ずるやむを得ない事由があったとき。
  \end{enumerate}
\subsubsection{}
\label{就規_項_解雇の予告}
前項の規定により従業員を解雇する場合は、少なくとも30日前に予告をする。予告しないときは、平均賃金の30日分以上の手当を解雇予告手当として支払う。ただし、予告の日数については、解雇予告手当を支払った日数だけ短縮することができる。
\subsubsection{}
\label{就規_項_懲戒解雇時の適用除外}
\ref{就規_項_解雇の予告}の規定は、労働基準監督署長の認定を受けて従業員を\ref{就規_条_懲戒の事由}に定める懲戒解雇する場合又は次の各号のいずれかに該当する従業員を解雇する場合は適用しない。
  \begin{enumerate}
    \item 日々雇い入れられる従業員(ただし、1か月を超えて引き続き使用されるに至った者を除く。)
    \item 2か月以内の期間を定めて使用する従業員(ただし、その期間を超えて引き続き使用されるに至った者を除く。)
    \item 試用期間中の従業員(ただし、14日を超えて引き続き使用されるに至った者を除く。)
  \end{enumerate}
\subsubsection{}
\label{就規_項_解雇理由の証明書交付}
\ref{就規_項_解雇}の規定による従業員の解雇に際して従業員から請求のあった場合は、解雇の理由を記載した証明書を交付する。

\section{退職金}
\label{就規_章_退職金}

\subsection{退職金}
\label{就規_条_退職金}
\subsubsection{}
\label{就規_項_退職金}
退職金は別に定める退職金規程の定めるところによる。

\section{安全衛生及び災害補償}
\label{就規_章_安全衛生及び災害補償}

\subsection{安全衛生}
\label{就規_条_安全衛生}
\subsubsection{}
\label{就規_項_安全衛生}
会社は、従業員の安全衛生の確保及び改善を図り、快適な職場の形成のために必要な措置を講ずる。
\subsubsection{}
\label{就規_項_労働災害防止の努力義務}
従業員は、安全衛生に関する法令及び会社の指示を守り、会社と協力して労働災害の防止に努めなければならない。
\subsubsection{}
\label{就規_項_安全衛生の確保}
従業員は安全衛生の確保のため、特に下記の事項を遵守しなければならない。
  \begin{enumerate}
    \item 機械設備、工具等の就業前点検を徹底すること。また、異常を認めたときは、速やかに会社に報告し、指示に従うこと。
    \item 安全装置を取り外したり、その効力を失わせるようなことはしないこと。
    \item 保護具の着用が必要な作業については、必ず着用すること。
    \item 喫煙は、所定の場所以外では行わないこと。
    \item 立入禁止又は通行禁止区域には立ち入らないこと。
    \item 常に整理整頓に努め、通路、避難口又は消火設備のある所に物品を置かないこと。
    \item 火災等非常災害の発生を発見したときは、直ちに臨機の措置をとり、上長に報告し、その指示に従うこと。
  \end{enumerate}

\subsection{健康診断}
\label{就規_条_健康診断}
\subsubsection{}
\label{就規_項_定期健康診断}
従業員に対しては、採用の際及び毎年1回(深夜労働に従事する者は6か月ごとに1回)、定期に健康診断を行う。
\subsubsection{}
\label{就規_項_特別健康診断}
\ref{就規_項_定期健康診断}の健康診断のほか、法令で定められた有害業務に従事する従業員に対しては、特別の項目についての健康診断を行う。
\subsubsection{}
\label{就規_項_医師の面接指導}
長時間の労働により疲労の蓄積が認められる従業員に対し、その者の申出により医師による面接指導を行う。
\subsubsection{}
\label{就規_項_健康保持上必要な措置}
\ref{就規_項_定期健康診断}及び\ref{就規_項_特別健康診断}の健康診断並びに前項の面接指導の結果必要と認めるときは、一定期間の就業禁止、労働時間の短縮、配置転換その他健康保持上必要な措置を命ずることがある。

\subsection{安全衛生教育}
\label{就規_条_安全衛生教育}
\subsubsection{}
\label{就規_項_安全衛生教育}
従業員に対し、雇入れの際及び配置換え等により作業内容を変更した場合、その従事する業務に必要な安全及び衛生に関する教育を行う。
\subsubsection{}
\label{就規_項_安全衛生教育指導事項の遵守}
従業員は、安全衛生教育を受けた事項を遵守しなければならない。

\subsection{災害補償}
\label{就規_条_災害補償}
\subsubsection{}
\label{就規_項_災害補償}
従業員が業務上の事由又は通勤により負傷し、疾病にかかり、又は死亡した場合は、労基法及び労働者災害補償保険法(昭和22年法律第50号)に定めるところにより災害補償を行う。

\section{表彰及び制裁}
\label{就規_章_表彰及び制裁}

\subsection{表彰}
\label{就規_条_表彰}
\subsubsection{}
\label{就規_項_表彰}
会社は、従業員が次のいずれかに該当するときは、表彰することがある。
  \begin{enumerate}
    \item 業務上有益な発明、考案を行い、会社の業績に貢献したとき。
    \item 永年にわたって誠実に勤務し、その成績が優秀で他の模範となるとき。
    \item 永年にわたり無事故で継続勤務したとき。
    \item 社会的功績があり、会社及び従業員の名誉となったとき。
    \item 前各号に準ずる善行又は功労のあったとき。
  \end{enumerate}
\subsubsection{}
\label{就規_項_賞品}
表彰は、賞状のほか賞金を授与する。

\subsection{懲戒の種類}
\label{就規_条_懲戒の種類}
\subsubsection{}
\label{就規_項_懲戒の種類}
会社は、従業員が次条のいずれかに該当する場合は、その情状に応じ、次の区分により懲戒を行う。
  \begin{enumerate}
    \item けん責 \\
          始末書を提出させて将来を戒める。
    \item 減給 \\
          始末書を提出させて減給する。ただし、減給は1回の額が平均賃金の1日分の5割を超えることはなく、また、総額が1賃金支払期における賃金総額の1割を超えることはない。
    \item 出勤停止 \\
          始末書を提出させるほか、10日間を限度として出勤を停止し、その間の賃金は支給しない。
    \item 懲戒解雇 \\
          予告期間を設けることなく即時に解雇する。この場合において、所轄の労働基準監督署長の認定を受けたときは、解雇予告手当(平均賃金の30日分)を支給しない。
  \end{enumerate}

\subsection{懲戒の事由}
\label{就規_条_懲戒の事由}
\subsubsection{}
\label{就規_項_懲戒の事由}
従業員が次のいずれかに該当するときは、情状に応じ、けん責、減給又は出勤停止とする。
  \begin{enumerate}
    \item 正当な理由なく無断欠勤が1日以上に及ぶとき。
    \item 正当な理由なくしばしば欠勤、遅刻、早退をしたとき。
    \item 過失により会社に損害を与えたとき。
    \item 素行不良で社内の秩序及び風紀を乱したとき。
    \item 性的な言動により、他の従業員に不快な思いをさせ、又は職場の環境を悪くしたとき。
    \item 性的な関心を示し、又は性的な行為をしかけることにより、他の従業員の業務に支障を与えたとき。
    \item \ref{就規_条_遵守事項}、\ref{就規_条_個人情報保護}に違反したとき。
    \item その他この規則に違反し又は前各号に準ずる不都合な行為があったとき。
  \end{enumerate}
\subsubsection{}
\label{就規_項_懲戒解雇}
従業員が次のいずれかに該当するときは、懲戒解雇とする。ただし、平素の服務態度その他情状によっては、\ref{就規_条_解雇}に定める普通解雇、前条に定める減給又は出勤停止とすることがある。
  \begin{enumerate}
    \item 重要な経歴を詐称して雇用されたとき。
    \item 正当な理由なく無断欠勤が14日以上に及び、出勤の督促に応じなかったとき。
    \item 正当な理由なく無断でしばしば遅刻、早退又は欠勤を繰り返し、3回にわたって注意を受けても改めなかったとき。
    \item 正当な理由なく、しばしば業務上の指示・命令に従わなかったとき。
    \item 故意又は重大な過失により会社に重大な損害を与えたとき。
    \item 会社内において刑法その他刑罰法規の各規定に違反する行為を行い、その犯罪事実が明らかとなったとき(当該行為が軽微な違反である場合を除く。)。
    \item 素行不良で著しく社内の秩序又は風紀を乱したとき。
    \item 数回にわたり懲戒を受けたにもかかわらず、なお、勤務態度等に関し、改善の見込みがないとき。
    \item 職責を利用して交際を強要し、又は性的な関係を強要したとき。
    \item 許可なく職務以外の目的で会社の施設、物品等を使用したとき。
    \item 職務上の地位を利用して私利を図り、又は取引先等より不当な金品を受け、若しくは求め若しくは供応を受けたとき。
    \item 私生活上の非違行為や会社に対する正当な理由のない誹謗中傷等であって、会社の名誉信用を損ない、業務に重大な悪影響を及ぼす行為をしたとき。
    \item 正当な理由なく会社の業務上重要な秘密を外部に漏洩して会社に損害を与え、又は業務の正常な運営を阻害したとき。
    \item その他前各号に準ずる不適切な行為があったとき。
  \end{enumerate}

\fusoku{}
\subsection{附則}
本規則は、平成◯年◯月◯日から適用する。

\end{document}
