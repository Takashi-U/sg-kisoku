\documentclass{jsarticle}
\usepackage{kisoku_base_macros}

\begin{document}
\title{育児介護休業規程}
\author{Takashi Uchibe}
\maketitle

\section{目的}

\subsection{目的}
本規則は、従業員の育児・介護休業、子の看護休暇、介護休暇、育児のための所定外労働の免除、育児・介護のための時間外労働及び深夜業の制限並びに育児・介護短時間勤務等に関する取扱いについて定めるものである。

\section{育児休業制度}
\label{育介_章_育児休業制度}

\subsection{育児休業の対象者}
\label{育介_条_育児休業の対象者}

\subsubsection{育児休業を取得できる者}
\label{育介_項_育児休業を取得できる者}
育児のために休業することを希望する従業員(日雇従業員を除く)であって、1歳に満たない子と同居し、養育する者は、この規則に定めるところにより育児休業をすることができる。ただし、期間契約従業員にあっては、申出時点において、次のいずれにも該当する者に限り育児休業をすることができる。
\begin{enumerate}
  \item 入社1年以上であること
  \item 子が1歳に達する日を超えて雇用関係が継続することが見込まれること
  \item 子が1歳に達する日から1年を経過する日までに労働契約期間が満了し、更新されないことが明らかでないこと
\end{enumerate}

\subsubsection{労使協定により育児休業取得者から除外される者}
\label{育介_項_労使協定により育児休業取得者から除外される者}
会社は\ref{育介_項_育児休業を取得できる者}にかかわらず、労使協定により除外された次の従業員からの休業の申出は拒むことができる。
\begin{enumerate}
  \item 入社1年未満の従業員
  \item 申出の日から1年以内に雇用関係が終了することが明らかな従業員
  \item 1週間の所定労働日数が2日以下の従業員
\end{enumerate}

\subsubsection{育児休業の期間}
\label{育介_項_育児休業の期間}
配偶者が従業員と同じ日から又は従業員より先に育児休業をしている場合、従業員は、子が1歳2か月に達するまでの間で、出生日以後の産前・産後休業期間と育児休業期間との合計が1年を限度として、育児休業をすることができる。

\subsubsection{育児休業の期間(特別な事情のある場合)}
\label{育介_項_育児休業の期間(特別な事情のある場合)}
次のいずれにも該当する従業員は、子が1歳6か月に達するまでの間で必要な日数について育児休業をすることができる。なお、育児休業を開始しようとする日は、原則として子の1歳の誕生日に限るものとする。
\begin{enumerate}
  \item 従業員又は配偶者が原則として子の1歳の誕生日の前日に育児休業をしていること
  \item 次のいずれかの事情があること
    \begin{enumerate}
      \item 保育所に入所を希望しているが、入所できない場合
      \item 従業員の配偶者であって育児休業の対象となる子の親であり、1歳以降育児に当たる予定であった者が、死亡、負傷、疾病等の事情により子を養育することが困難になった場合
    \end{enumerate}
\end{enumerate}

\subsection{育児休業の申出の手続等}
\label{育介_条_育児休業の申出の手続等}

\subsubsection{育児休業の申出}
\label{育介_項_育児休業の申出}
育児休業をすることを希望する従業員は、原則として育児休業を開始しようとする日(以下「育児休業開始予定日」という。)の1か月前(\ref{育介_項_育児休業の期間(特別な事情のある場合)}に基づく1歳を超える休業の場合は、2週間前)までに育児休業申出書(社内様式1)を人事部労務課に提出することにより申し出るものとする。なお、育児休業中の期間契約従業員が労働契約を更新するに当たり、引き続き休業を希望する場合には、更新された労働契約期間の初日を育児休業開始予定日として、育児休業申出書により再度の申出を行うものとする。 

\subsubsection{育児休業の回数}
\label{育介_項_育児休業の回数}
申出は、次のいずれかに該当する場合を除き、一子につき1回限りとする。ただし、産後休業をしていない従業員が、子の出生日又は出産予定日のいずれか遅い方から8週間以内にした最初の育児休業については、1回の申出にカウントしない。
\begin{enumerate}
  \item \ref{育介_項_育児休業を取得できる者}に基づく休業をした者が\ref{育介_項_育児休業の期間(特別な事情のある場合)}に基づく休業の申出をしようとする場合又は\ref{育介_項_育児休業の申出}後段の申出をしようとする場合
  \item 配偶者の死亡等特別の事情がある場合 
\end{enumerate}

\subsubsection{育児休業の申出時に必要な添付書類}
\label{育介_項_育児休業の申出時に必要な添付書類}
会社は、育児休業申出書を受け取るに当たり、必要最小限度の各種証明書の提出を求めることがある。

\subsubsection{育児休業申出時の取扱通知書の交付}
\label{育介_項_育児休業申出時の取扱通知書の交付}
育児休業申出書が提出されたときは、会社は速やかに当該育児休業申出書を提出した者(以下この章において「申出者」という。)に対し、育児休業取扱通知書(社内様式2)を交付する。

\subsubsection{育児休業対象児出生届}
\label{育介_項_育児休業対象児出生届}
申出の日後に申出に係る子が出生したときは、申出者は、出生後2週間以内に人事部労務課に育児休業対象児出生届(社内様式3)を提出しなければならない。

\subsection{育児休業の申出の撤回等}
\label{育介_条_育児休業の申出の撤回等}

\subsubsection{育児休業の申出の撤回}
\label{育介_項_育児休業の申出の撤回}
申出者は、育児休業開始予定日の前日までは、育児休業申出撤回届(社内様式4)を人事部労務課に提出することにより、育児休業の申出を撤回することができる。 

\subsubsection{育児休業申出撤回時の取扱通知書の交付}
\label{育介_項_育児休業申出撤回時の取扱通知書の交付}
育児休業申出撤回届が提出されたときは、会社は速やかに当該育児休業申出撤回届を提出した者に対し、育児休業取扱通知書(社内様式2)を交付する。

\subsubsection{育児休業申出撤回時の再申請}
\label{育介_項_育児休業申出撤回時の再申請}
育児休業の申出を撤回した者は、特別の事情がない限り同一の子については再度申出をすることができない。ただし、\ref{育介_項_育児休業を取得できる者}に基づく休業の申出を撤回した者であっても、\ref{育介_項_育児休業の期間(特別な事情のある場合)}に基づく休業の申出をすることができる。

\subsubsection{育児休業申出に係る子を養育しないこととなった場合}
\label{育介_項_育児休業申出に係る子を養育しないこととなった場合}
育児休業開始予定日の前日までに、子の死亡等により申出者が休業申出に係る子を養育しないこととなった場合には、育児休業の申出はされなかったものとみなす。この場合において、申出者は、原則として当該事由が発生した日に、人事部労務課にその旨を通知しなければならない。

\subsection{育児休業の期間等}
\label{育介_条_育児休業の期間等}

\subsubsection{育児休業期間の申出}
\label{育介_項_育児休業期間の申出}
育児休業の期間は、原則として、子が1歳に達するまで(\ref{育介_項_育児休業の期間}及び\ref{育介_項_育児休業の期間(特別な事情のある場合)})に基づく休業の場合は、それぞれ定められた時期まで)を限度として育児休業申出書(社内様式1)に記載された期間とする。 

\subsubsection{育児休業期間の指定}
\label{育介_項_育児休業期間の指定}
\ref{育介_項_育児休業期間の申出}にかかわらず、会社は、育児・介護休業法の定めるところにより育児休業開始予定日の指定を行うことができる。

\subsubsection{育児休業期間変更の申出}
\label{育介_項_育児休業期間変更の申出}
従業員は、育児休業期間変更申出書(社内様式5)により人事部労務課に、育児休業開始予定日の1週間前までに申し出ることにより、育児休業開始予定日の繰り上げ変更を、また、育児休業を終了しようとする日(以下「育児休業終了予定日」という。)の1か月前(\ref{育介_項_育児休業の期間(特別な事情のある場合)}に基づく休業をしている場合は、2週間前)までに申し出ることにより、育児休業終了予定日の繰り下げ変更を行うことができる。
育児休業開始予定日の繰り上げ変更及び育児休業終了予定日の繰り下げ変更とも、原則として1回に限り行うことができるが、\ref{育介_項_育児休業の期間(特別な事情のある場合)}に基づく休業の場合には、\ref{育介_項_育児休業を取得できる者}に基づく休業とは別に、子が1歳から1歳6か月に達するまでの期間内で、一回、育児休業終了予定日の繰り下げ変更を行うことができる。 

\subsubsection{育児休業期間変更時の取扱通知書の交付}
\label{育介_項_育児休業期間変更時の取扱通知書の交付}
育児休業期間変更申出書が提出されたときは、会社は速やかに当該育児休業期間変更申出書を提出した者に対し、育児休業取扱通知書(社内様式2)を交付する。

\subsubsection{育児休業の終了}
\label{育介_項_育児休業の終了}
次の各号に掲げるいずれかの事由が生じた場合には、育児休業は終了するものとし、当該育児休業の終了日は当該各号に掲げる日とする。 
\begin{enumerate}
  \item 子の死亡等育児休業に係る子を養育しないこととなった場合\label{enum:育介_項_育児休業の終了_子の死亡等}
         当該事由が発生した日(なお、この場合において本人が出勤する日は、事由発生の日から2週間以内であって、会社と本人が話し合いの上決定した日とする。)
  \item 育児休業に係る子が1歳に達した場合等
         子が1歳に達した日(\ref{育介_項_育児休業の期間}に基づく休業の場合を除く。\ref{育介_項_育児休業の期間(特別な事情のある場合)}に基づく休業の場合は、子が1歳6か月に達した日) 
  \item 申出者について、産前産後休業、介護休業又は新たな育児休業期間が始まった場合
         産前産後休業、介護休業又は新たな育児休業の開始日の前日
  \item \ref{育介_項_育児休業の期間}に基づく休業において、出生日以後の産前・産後休業期間と育児休業期間との合計が1年に達した場合
         当該1年に達した日
\end{enumerate}

\subsubsection{育児休業期間中に子を養育しないこととなった場合}
\label{育介_項_育児休業期間中に子を養育しないこととなった場合}
\ref{育介_項_育児休業の終了}\ref{enum:育介_項_育児休業の終了_子の死亡等}の事由が生じた場合には、申出者は原則として当該事由が生じた日に人事部労務課にその旨を通知しなければならない。 

\section{介護休業制度}

\subsection{介護休業の対象者}
\label{育介_条_介護休業の対象者}

\subsubsection{介護休業をすることができる者}
\label{育介_項_介護休業をすることができる者}
要介護状態にある家族を介護する従業員(日雇従業員を除く)は、この規則に定めるところにより介護休業をすることができる。ただし、期間契約従業員にあっては、申出時点において、次のいずれにも該当する者に限り介護休業をすることができる。
\begin{enumerate}
  \item 入社1年以上であること。
  \item 介護休業を開始しようとする日(以下「介護休業開始予定日」という。)から 93日を経過する日(93日経過日)を超えて雇用関係が継続することが見込まれること。
  \item 93日経過日から1年を経過する日までに労働契約期間が満了し、更新されないことが明らかでないこと。 
\end{enumerate}

\subsubsection{労使協定により介護休業取得者から除外される者}
\label{育介_項_労使協定により介護休業取得者から除外される者}
\ref{育介_項_介護休業をすることができる者}にかかわらず、労使協定により除外された次の従業員からの休業の申出は拒むことができる。
\begin{enumerate}
  \item 入社1年未満の従業員
  \item 申出の日から93日以内に雇用関係が終了することが明らかな従業員
  \item 1週間の所定労働日数が2日以下の従業員
\end{enumerate}

\subsubsection{要介護状態にある家族}
\label{育介_項_要介護状態にある家族}
要介護状態にある家族とは、負傷、疾病又は身体上若しくは精神上の障害により、2週間以上の期間にわたり常時介護を必要とする状態にある次の者をいう。 
\begin{enumerate}
  \item 配偶者
  \item 父母
  \item 子 
  \item 配偶者の父母 
  \item 祖父母、兄弟姉妹又は孫であって従業員が同居し、かつ、扶養している者 
  \item 上記以外の家族で会社が認めた者
\end{enumerate}

\subsection{介護休業の申出の手続等}
\label{育介_条_介護休業の申出の手続等}

\subsubsection{介護休業の申出}
\label{育介_項_介護休業の申出}
介護休業をすることを希望する従業員は、原則として介護休業開始予定日の2週間前までに、介護休業申出書(社内様式6)を人事部労務課に提出することにより申し出るものとする。なお、介護休業中の期間契約従業員が労働契約を更新するに当たり、引き続き休業を希望する場合には、更新された労働契約期間の初日を介護休業開始予定日として、介護休業申出書により再度の申出を行うものとする。

\subsubsection{介護休業の回数}
\label{育介_項_介護休業の回数}
申出は、特別な事情がない限り、対象家族1人につき1要介護状態ごとに1回とする。ただし、1の後段の申出をしようとする場合にあっては、この限りでない。

\subsubsection{介護休業の申出時に必要な添付書類}
\label{育介_項_介護休業の申出時に必要な添付書類}
会社は、介護休業申出書を受け取るに当たり、必要最小限度の各種証明書の提出を求めることがある。

\subsubsection{介護休業申出時の取扱通知書の交付}
\label{育介_項_介護休業申出時の取扱通知書の交付}
介護休業申出書が提出されたときは、会社は速やかに当該介護休業申出書を提出した者(以下この章において「申出者」という。)に対し、介護休業取扱通知書(社内様式2)を交付する。

\subsection{介護休業の申出の撤回等}
\label{育介_条_介護休業の申出の撤回等}

\subsubsection{介護休業の申出の撤回}
\label{育介_項_介護休業の申出の撤回}
申出者は、介護休業開始予定日の前日までは、介護休業申出撤回届(社内様式4)を人事部労務課に提出することにより、介護休業の申出を撤回することができる。

\subsubsection{介護休業申出撤回時の取扱通知書の交付}
\label{育介_項_介護休業申出撤回時の取扱通知書の交付}
介護休業申出撤回届が提出されたときは、会社は速やかに当該介護休業申出撤回届を提出した者に対し、介護休業取扱通知書(社内様式2)を交付する。

\subsubsection{介護休業申出撤回時の再申請}
\label{育介_項_介護休業申出撤回時の再申請}
介護休業の申出を撤回した者について、同一対象家族の同一要介護状態に係る再度の申出は原則として1回とし、特段の事情がある場合について会社がこれを適当と認めた場合には、1回を超えて申し出ることができるものとする。 

\subsubsection{介護休業申出に係る家族を介護しないこととなった場合}
\label{育介_項_介護休業申出に係る家族を介護しないこととなった場合}
介護休業開始予定日の前日までに、申出に係る家族の死亡等により申出者が家族を介護しないこととなった場合には、介護休業の申出はされなかったものとみなす。この場合において、申出者は、原則として当該事由が発生した日に、人事部労務課にその旨を通知しなければならない。

\subsection{介護休業の期間等}
\label{育介_条_介護休業の期間等}

\subsubsection{介護休業の期間の申出}
\label{育介_項_介護休業の期間の申出}
介護休業の期間は、対象家族1人につき、原則として、通算93日間の範囲(介護休業開始予定日から起算して93日を経過する日までをいう。)内で、介護休業申出書(社内様式6)に記載された期間とする。
ただし、同一家族について、異なる要介護状態について介護休業をしたことがある場合又は\ref{育介_条_介護短時間勤務}に規定する介護短時間勤務の適用を受けた場合は、その日数も通算して93日間までを原則とする。

\subsubsection{介護休業期間の指定}
\label{育介_項_介護休業期間の指定}
\ref{育介_項_介護休業の期間の申出}にかかわらず、会社は、育児・介護休業法の定めるところにより介護休業開始予定日の指定を行うことができる。 

\subsubsection{介護休業期間変更の申出}
\label{育介_項_介護休業期間変更の申出}
従業員は、介護休業期間変更申出書(社内様式5)により、介護休業を終了しようとする日(以下「介護休業終了予定日」という。)の2週間前までに人事部労務課に申し出ることにより、介護休業終了予定日の繰下げ変更を行うことができる。
この場合において、介護休業開始予定日から変更後の介護休業終了予定日までの期間は通算93日(異なる要介護状態について介護休業をしたことがある場合又は\ref{育介_条_介護短時間勤務}に規定する介護短時間勤務の適用を受けた場合は、93日からその日数を控除した日数)の範囲を超えないことを原則とする。 

\subsubsection{介護休業期間変更時の取扱通知書の交付}
\label{育介_項_介護休業期間変更時の取扱通知書の交付}
介護休業期間変更申出書が提出されたときは、会社は速やかに当該介護休業期間変更申出書を提出した者に対し、介護休業取扱通知書(社内様式2)を交付する。 

\subsubsection{介護休業の終了}
\label{育介_項_介護休業の終了}
次の各号に掲げるいずれかの事由が生じた場合には、介護休業は終了するものとし、当該介護休業の終了日は当該各号に掲げる日とする。 
\begin{enumerate}
  \item 家族の死亡等介護休業に係る家族を介護しないこととなった場合\label{enum:育介_項_介護休業の終了_家族の死亡等}
         当該事由が発生した日(なお、この場合において本人が出勤する日は、事由発生の日から2週間以内であって、会社と本人が話し合いの上決定した日とする。)
  \item 申出者について、産前産後休業、育児休業又は新たな介護休業が始まった場合
         産前産後休業、育児休業又は新たな介護休業の開始日の前日
\end{enumerate}

\subsubsection{介護休業期間中に家族を介護しないこととなった場合}
\label{育介_項_介護休業期間中に家族を介護しないこととなった場合}
\ref{育介_項_介護休業の終了}\ref{enum:育介_項_介護休業の終了_家族の死亡等}の事由が生じた場合には、申出者は原則として当該事由が生じた日に人事部労務課にその旨を通知しなければならない。 

\section{子の看護休暇}

\subsection{子の看護休暇}
\label{育介_条_子の看護休暇}

\subsubsection{子の看護休暇を取得できる者}
\label{育介_項_子の看護休暇を取得できる者}
小学校就学の始期に達するまでの子を養育する従業員(日雇従業員を除く)は、負傷し、又は疾病にかかった当該子の世話をするために、又は当該子に予防接種や健康診断を受けさせるために、就業規則に定める年次有給休暇とは別に、当該子が1人の場合は1年間につき5日、2人以上の場合は1年間につき10日を限度として、子の看護休暇を取得することができる。この場合の1年間とは、4月1日から翌年3月31日までの期間とする。ただし、労使協定によって除外された次の従業員からの子の看護休暇の申出は拒むことができる。
\begin{enumerate}
  \item 入社6か月未満の従業員
  \item 1週間の所定労働日数が2日以下の従業員 
\end{enumerate}

\subsubsection{時間単位の子の看護休暇取得}
\label{育介_項_時間単位の子の看護休暇取得}
子の看護休暇は、時間単位で取得することができる。

\subsubsection{子の看護休暇取得の申出}
\label{育介_項_子の看護休暇取得の申出}
取得しようとする者は、原則として、事前に人事部労務課に申し出るものとする。 

\subsubsection{子の看護休暇中の給与・賞与・昇給・退職金の取扱}
\label{育介_項_子の看護休暇中の給与・賞与・昇給・退職金の取扱}
給与、賞与、定期昇給及び退職金の算定に当たっては、取得期間は通常の勤務をしたものとみなす。

\section{介護休暇}

\subsection{介護休暇}
\label{育介_条_介護休暇}

\subsubsection{介護休暇を取得できる者}
\label{育介_項_介護休暇を取得できる者}
要介護状態にある家族の介護その他の世話をする従業員(日雇従業員を除く)は、就業規則に定める年次有給休暇とは別に、当該対象家族が1人の場合は1年間につき5日、2人以上の場合は1年間につき10日を限度として、介護休暇を取得することができる。この場合の1年間とは、4月1日から翌年3月31日までの期間とする。ただし、労使協定によって除外された次の従業員からの介護休暇の申出は拒むことができる。
\begin{enumerate}
  \item 入社6か月未満の従業員
  \item 1週間の所定労働日数が2日以下の従業員 
\end{enumerate}

\subsubsection{時間単位の介護休暇取得}
\label{育介_項_時間単位の介護休暇取得}
介護休暇は、時間単位で取得することができる。

\subsubsection{介護休暇取得の申出}
\label{育介_項_介護休暇取得の申出}
取得しようとする者は、原則として、事前に人事部労務課に申し出るものとする。 

\subsubsection{介護休暇中の給与・賞与・昇給・退職金の取扱}
\label{育介_項_介護休暇中の給与・賞与・昇給・退職金の取扱}
給与、賞与、定期昇給及び退職金の算定に当たっては、取得期間は通常の勤務をしたものとみなす。

\section{所定外労働の免除}

\subsection{育児のための所定外労働の免除}
\label{育介_条_育児のための所定外労働の免除}

\subsubsection{所定外労働の免除措置を受けられる者}
\label{育介_項_所定外労働の免除措置を受けられる者}
3歳に満たない子を養育する従業員(日雇従業員を除く)が当該子を養育するために申し出た場合には、事業の正常な運営に支障がある場合を除き、所定労働時間を超えて労働をさせることはない。 

\subsubsection{労使協定により所定外労働の免除措置を受けられる者から除外される者}
\label{育介_項_労使協定により所定外労働の免除措置を受けられる者から除外される者}
\ref{育介_項_所定外労働の免除措置を受けられる者}にかかわらず、労使協定によって除外された次の従業員からの所定外労働の免除の申出は拒むことができる。 
\begin{enumerate}
  \item 入社1年未満の従業員
  \item 1週間の所定労働日数が2日以下の従業員
\end{enumerate}

\subsubsection{所定外労働の免除措置の申出}
\label{育介_項_所定外労働の免除措置の申出}
申出をしようとする者は、1回につき、1か月以上1年以内の期間(以下この条において「免除期間」という。)について、免除を開始しようとする日(以下この条において「免除開始予定日」という。)及び免除を終了しようとする日を明らかにして、原則として、免除開始予定日の1か月前までに、育児のための所定外労働免除申出書(社内様式7)を人事部労務課に提出するものとする。この場合において、免除期間は、\ref{育介_項_育児・介護のための時間外労働の制限の申出}に規定する制限期間と重複しないようにしなければならない。

\subsubsection{所定外労働の免除措置の申出時に必要な添付書類}
\label{育介_項_所定外労働の免除措置の申出時に必要な添付書類}
会社は、所定外労働免除申出書を受け取るに当たり、必要最小限度の各種証明書の提出を求めることがある。 

\subsubsection{所定外労働免除対象児出生届}
\label{育介_項_所定外労働免除対象児出生届}
申出の日後に申出に係る子が出生したときは、所定外労働免除申出書を提出した者(以下この条において「申出者」という。)は、出生後2週間以内に人事部労務課に所定外労働免除対象児出生届(社内様式3)を提出しなければならない。 

\subsubsection{所定外労働の免除措置に係る子を養育しないこととなった場合}
\label{育介_項_所定外労働の免除措置に係る子を養育しないこととなった場合}
免除開始予定日の前日までに、申出に係る子の死亡等により申出者が子を養育しないこととなった場合には、申出されなかったものとみなす。この場合において、申出者は、原則として当該事由が発生した日に、人事部労務課にその旨を通知しなければならない。

\subsubsection{所定外労働の免除措置の終了}
\label{育介_項_所定外労働の免除措置の終了}
次の各号に掲げるいずれかの事由が生じた場合には、免除期間は終了するものとし、当該免除期間の終了日は当該各号に掲げる日とする。 
\begin{enumerate}
  \item 子の死亡等免除に係る子を養育しないこととなった場合 当該事由が発生した日 
  \item 免除に係る子が3歳に達した場合 当該3歳に達した日 
  \item 申出者について、産前産後休業、育児休業又は介護休業が始まった場合 産前産後休業、育児休業又は介護休業の開始日の前日 
\end{enumerate}

\subsubsection{所定外労働の免除措置中に子を養育しないこととなった場合}
\label{育介_項_所定外労働の免除措置中に子を養育しないこととなった場合}
\ref{育介_項_所定外労働の免除措置の終了}の事由が生じた場合には、申出者は原則として当該事由が生じた日に、人事部労務課にその旨を通知しなければならない。 

\section{時間外労働の制限}

\subsection{育児・介護のための時間外労働の制限}
\label{育介_条_育児・介護のための時間外労働の制限}

\subsubsection{育児・介護のための時間外労働の制限措置を受けられる者} 
\label{育介_項_育児・介護のための時間外労働の制限措置を受けられる者} 
小学校就学の始期に達するまでの子を養育する従業員が当該子を養育するため又は要介護状態にある家族を介護する従業員が当該家族を介護するために申し出た場合には、就業規則の規定及び時間外労働に関する協定にかかわらず、事業の正常な運営に支障がある場合を除き、1か月について24時間、1年について150時間を超えて時間外労働をさせることはない。

\subsubsection{労使協定により時間外労働の制限措置を受けられる者から除外される者}
\label{育介_項_労使協定により時間外労働の制限措置を受けられる者から除外される者}
\ref{育介_項_育児・介護のための時間外労働の制限措置を受けられる者} にかかわらず、次の各号のいずれかに該当する従業員からの時間外労働の制限の申出は拒むことができる。
\begin{enumerate}
  \item 日雇従業員
  \item 入社1年未満の従業員 
  \item 1週間の所定労働日数が2日以下の従業員 
\end{enumerate}

\subsubsection{育児・介護のための時間外労働の制限の申出}
\label{育介_項_育児・介護のための時間外労働の制限の申出}
申出をしようとする者は、1回につき、1か月以上1年以内の期間(以下この条において「制限期間」という。)について、制限を開始しようとする日(以下この条において「制限開始予定日」という。)及び制限を終了しようとする日を明らかにして、原則として、制限開始予定日の1か月前までに、育児・介護のための時間外労働制限申出書(社内様式8)を人事部労務課に提出するものとする。 この場合において、制限期間は、前条第2項に規定する免除期間と重複しないようにしなければならない。

\subsubsection{時間外労働の免除措置の申出時に必要な添付書類}
\label{育介_項_時間外労働の免除措置の申出時に必要な添付書類}
会社は、時間外労働制限申出書を受け取るに当たり、必要最小限度の各種証明書の提出を求めることがある。 

\subsubsection{時間外労働制限対象児出生届}
\label{育介_項_時間外労働制限対象児出生届}
申出の日後に申出に係る子が出生したときは、時間外労働制限申出書を提出した者(以下この条において「申出者」という。)は、出生後2週間以内に人事部労務課に時間外労働制限対象児出生届(社内様式3)を提出しなければならない。 

\subsubsection{時間外労働の免除措置に係る子を養育又は家族を介護しないこととなった場合}
\label{育介_項_時間外労働の免除措置に係る子を養育又は家族を介護しないこととなった場合}
制限開始予定日の前日までに、申出に係る家族の死亡等により申出者が子を養育又は家族を介護しないこととなった場合には、申出されなかったものとみなす。この場合において、申出者は、原則として当該事由が発生した日に、人事部労務課にその旨を通知しなければならない。

\subsubsection{時間外労働の免除措置の終了}
\label{育介_項_時間外労働の免除措置の終了}
次の各号に掲げるいずれかの事由が生じた場合には、制限期間は終了するものとし、当該制限期間の終了日は当該各号に掲げる日とする。 
\begin{enumerate}
  \item 家族の死亡等制限に係る子を養育又は家族を介護しないこととなった場合\label{enum:育介_項_時間外労働の免除措置の終了_子・家族の死亡等}
         当該事由が発生した日 
  \item 制限に係る子が小学校就学の始期に達した場合
         子が6歳に達する日の属する年度の3月31日 
  \item 申出者について、産前産後休業、育児休業又は介護休業が始まった場合
         産前産後休業、育児休業又は介護休業の開始日の前日 
\end{enumerate}

\subsubsection{時間外労働の免除措置中に子を養育又は家族を介護しないこととなった場合}
\label{育介_項_時間外労働の免除措置中に子を養育又は家族を介護しないこととなった場合}
\ref{育介_項_時間外労働の免除措置の終了}\ref{enum:育介_項_時間外労働の免除措置の終了_子・家族の死亡等}の事由が生じた場合には、申出者は原則として当該事由が生じた日に、人事部労務課にその旨を通知しなければならない。

\section{深夜業の制限}

\subsection{育児・介護のための深夜業の制限}
\label{育介_条_育児・介護のための深夜業の制限}

\subsubsection{育児・介護のための深夜業の制限措置を受けられる者}
\label{育介_項_育児・介護のための深夜業の制限措置を受けられる者}
小学校就学の始期に達するまでの子を養育する従業員が当該子を養育するため又は要介護状態にある家族を介護する従業員が当該家族を介護するために申し出た場合には、就業規則の定めにかかわらず、事業の正常な運営に支障がある場合を除き、午後10時から午前5時までの間(以下「深夜」という。)に労働させることはない。 

\subsubsection{労使協定により育児・介護のための深夜業の制限措置を受けられる者から除外される者}
\label{育介_項_労使協定により育児・介護のための深夜業の制限措置を受けられる者から除外される者}
\ref{育介_項_育児・介護のための深夜業の制限措置を受けられる者}にかかわらず、次のいずれかに該当する従業員からの深夜業の制限の申出は拒むことができる。
\begin{enumerate}
  \item 日雇従業員 
  \item 入社1年未満の従業員 
  \item 申出に係る家族の16歳以上の同居の家族が次のいずれにも該当する従業員
  \begin{enumerate}
    \item 深夜において就業していない者(1か月について深夜における就業が3日以下の者を含む。)であること。
    \item 心身の状況が申出に係る子の保育又は家族の介護をすることができる者であること。 
    \item 6週間(多胎妊娠の場合にあっては、14週間)以内に出産予定でなく、かつ産後8週間以内でない者であること。 
  \end{enumerate}
  \item 1週間の所定労働日数が2日以下の従業員
  \item 所定労働時間の全部が深夜にある従業員 
\end{enumerate}

\subsubsection{育児・介護のための深夜業の制限の申出}
\label{育介_項_育児・介護のための深夜業の制限の制限の申出}
申出をしようとする者は、1回につき、1か月以上6か月以内の期間(以下この条において「制限期間」という。)について、制限を開始しようとする日(以下この条において「制限開始予定日」という。)及び制限を終了しようとする日を明らかにして、原則として、制限開始予定日の1か月前までに、育児・介護のための深夜業制限申出書(社内様式9)を人事部労務課に提出するものとする。 

\subsubsection{育児・介護のための深夜業の制限の制限の申出時に必要な添付書類}
\label{育介_項_育児・介護のための深夜業の制限の制限の申出時に必要な添付書類}
会社は、深夜業制限申出書を受け取るに当たり、必要最小限度の各種証明書の提出を求めることがある。

\subsubsection{深夜業制限対象児出生届}
\label{育介_項_深夜業制限対象児出生届}
申出の日後に申出に係る子が出生したときは、深夜業制限申出書を提出した者(以下この条において「申出者」という。)は、出生後2週間以内に人事部労務課に深夜業制限対象児出生届(社内様式3)を提出しなければならない。 

\subsubsection{育児・介護のための深夜業の制限措置に係る子を養育又は家族を介護しないこととなった場合}
\label{育介_項_育児・介護のための深夜業の制限措置に係る子を養育又は家族を介護しないこととなった場合}
制限開始予定日の前日までに、申出に係る家族の死亡等により申出者が子を養育又は家族を介護しないこととなった場合には、申出されなかったものとみなす。この場合において、申出者は、原則として当該事由が発生した日に、人事部労務課にその旨を通知しなければならない。

\subsubsection{育児・介護のための深夜業の制限措置の終了}
\label{育介_項_育児・介護のための深夜業の制限措置の終了}
次の各号に掲げるいずれかの事由が生じた場合には、制限期間は終了するものとし、当該制限期間の終了日は当該各号に掲げる日とする。
\begin{enumerate}
  \item 家族の死亡等制限に係る子を養育又は家族を介護しないこととなった場合\label{enum:育介_項_育児・介護のための深夜業の制限措置の終了_子・家族の死亡等}
         当該事由が発生した日 
  \item 制限に係る子が小学校就学の始期に達した場合
         子が6歳に達する日の属する年度の3月31日 
  \item 申出者について、産前産後休業、育児休業又は介護休業が始まった場合
         産前産後休業、育児休業又は介護休業の開始日の前日
\end{enumerate}

\subsubsection{育児・介護のための深夜業の制限措置中に子を養育又は家族を介護しないこととなった場合}
\label{育介_項_育児・介護のための深夜業の制限措置中に子を養育又は家族を介護しないこととなった場合}
\ref{育介_項_時間外労働の免除措置の終了}\ref{enum:育介_項_育児・介護のための深夜業の制限措置の終了_子・家族の死亡等}の事由が生じた場合には、申出者は原則として当該事由が生じた日に、人事部労務課にその旨を通知しなければならない。

\subsubsection{育児・介護のための深夜業の制限措置中の給与}
\label{育介_項_育児・介護のための深夜業の制限措置中の給与}
制限期間中の給与については、別途定める給与規定に基づく基本給を時間換算した額を基礎とした実労働時間分の基本給と諸手当の全額を支給する。
\subsubsection{育児・介護のための深夜業の制限措置中の勤務時間帯変更}
\label{育介_項_育児・介護のための深夜業の制限措置中の勤務時間帯変更}
深夜業の制限を受ける従業員に対して、会社は必要に応じて昼間勤務ヘ転換させることがある。 

\section{所定労働時間の短縮措置等}

\subsection{育児短時間勤務}
\label{育介_条_育児短時間勤務}

\subsubsection{育児短時間勤務の措置を受けられる者}
\label{育介_項_育児短時間勤務の措置を受けられる者}
3歳に満たない子を養育する従業員は、申し出ることにより、就業規則定める所定労働時間について6時間に短縮する措置を受けることができる。始業・終業等の時刻は次の通りとする。
\begin{enumerate}
  \item[始業]午前9時
  \item[終業]午後4時
  \item[休憩]午前12時〜午後1時(1歳に満たない子を育てる女性従業員は更に別途30分ずつ2回の育児時間を請求することができる。)
\end{enumerate}
上記時刻は労使協議に基づき変更することがある。

\subsubsection{労使協定により育児短時間勤務の措置を受けられる者から除外される者}
\label{育介_項_労使協定により育児短時間勤務の措置を受けられる者から除外される者}
\ref{育介_項_育児短時間勤務の措置を受けられる者}にかかわらず、次のいずれかに該当する従業員からの育児短時間勤務の申出は拒むことができる。
\begin{enumerate}
  \item 日雇従業員
  \item 1日の所定労働時間が6時間以下である従業員
  \item 労使協定によって除外された次の従業員
    \begin{enumerate}
      \item 入社1年未満の従業員
      \item 1週間の所定労働日数が2日以下の従業員
      \item 業務の性質又は業務の実施体制に照らして所定労働時間の短縮措置を講ずることが困難と認められる業務として別に定める業務に従事する従業員\label{enum:育介_項_労使協定により育児短時間勤務の措置を受けられる者から除外される者_困難者}
    \end{enumerate}
\end{enumerate}

\subsubsection{育児短時間勤務の措置の手続}
\label{育介_項_育児短時間勤務の措置の手続}
申出をしようとする者は、1回につき、1か月以上1年以内の期間について、短縮を開始しようとする日及び短縮を終了しようとする日を明らかにして、原則として、短縮開始予定日の 1か月前までに、育児短時間勤務申出書(社内様式11)により人事部労務課に申し出なければならない。申出書が提出されたときは、会社は速やかに申出者に対し、育児短時間勤務取扱通知書(社内様式13)を交付する。その他適用のための手続等については、\ref{育介_条_育児休業の申出の手続等}から\ref{育介_条_育児休業の期間等}までの規定(\ref{育介_項_育児休業の回数}及び\ref{育介_項_育児休業申出撤回時の再申請}を除く。)を準用する。

\subsubsection{育児短時間勤務の措置中の給与}
\label{育介_項_育児短時間勤務の措置中の給与}
本制度の適用を受ける間の給与については、別途定める給与規定に基づく基本給を時間換算した額を基礎とした実労働時間分の基本給と諸手当の全額を支給する。

\subsubsection{育児短時間勤務の措置中の賞与}
\label{育介_項_育児短時間勤務の措置中の賞与}
賞与については、その算定対象期間に本制度の適用を受ける期間がある場合においては、短縮した時間に対応する賞与は支給しない。

\subsubsection{育児短時間勤務の措置中の昇給・退職金}
\label{育介_項_育児短時間勤務の措置中の昇給・退職金}
定期昇給及び退職金の算定に当たっては、本制度の適用を受ける期間は通常の勤務をしているものとみなす。

\subsection{業務上育児短時間勤務が困難な従業員に対する代替措置}
\label{育介_条_業務上育児短時間勤務が困難な従業員に対する代替措置}

\subsubsection{業務上育児短時間勤務が困難な従業員に対する代替措置}
\label{育介_項_業務上育児短時間勤務が困難な従業員に対する代替措置}
\ref{育介_項_労使協定により育児短時間勤務の措置を受けられる者から除外される者}\ref{enum:育介_項_労使協定により育児短時間勤務の措置を受けられる者から除外される者_困難者}の従業員は、申し出ることにより、子が3歳に達するまでの間、育児休業をすることができる。

\subsubsection{業務上育児短時間勤務が困難な従業員に対する代替措置の手続}
\label{育介_項_業務上育児短時間勤務が困難な従業員に対する代替措置の手続}
\ref{育介_項_業務上育児短時間勤務が困難な従業員に対する代替措置}の育児休業に関する手続その他の事項については、\ref{育介_章_育児休業制度}に定める育児休業に準じるものとする。ただし、\ref{育介_項_育児休業の回数}及び\ref{育介_項_育児休業申出撤回時の再申請}の規定は準用しない。

\subsection{介護短時間勤務}
\label{育介_条_介護短時間勤務}

\subsubsection{介護短時間勤務の措置を受けられる者}
\label{育介_項_介護短時間勤務の措置を受けられる者}
要介護状態にある家族を介護する従業員は、申し出ることにより、対象家族1人当たり通算93日間の範囲内を原則(ただし、同一家族について既に介護休業をした場合又は異なる要介護状態について介護短時間勤務の適用を受けた場合は、その日数も通算して93日間までを原則とする。)として、就業規則に定める所定労働時間について6時間に短縮する措置を受けることができる。始業・終業等の時刻は次の通りとする。
\begin{description}
  \item[始業]午前9時
  \item[終業]午後4時
  \item[休憩]午前12時〜午後1時(1歳に満たない子を育てる女性従業員は更に別途30分ずつ2回の育児時間を請求することができる。)
\end{description}
上記時刻は労使協議に基づき変更することがある。

\subsubsection{介護短時間勤務の措置を受けられる者から除外される者}
\label{育介_項_介護短時間勤務の措置を受けられる者から除外される者}
\ref{育介_項_介護短時間勤務の措置を受けられる者}にかかわらず、日雇従業員からの介護短時間勤務の申出は拒むことができる。

\subsubsection{介護短時間勤務の措置の手続}
\label{育介_項_介護短時間勤務の措置の手続}
申出をしようとする者は、1回につき、93日(介護休業をした場合又は異なる要介護状態について介護短時間勤務の適用を受けた場合は、93日からその日数を控除した日数)以内の期間について、短縮を開始しようとする日及び短縮を終了しようとする日を明らかにして、原則として、短縮開始予定日の2週間前までに、介護短時間勤務申出書(社内様式12)により人事部労務課に申し出なければならない。申出書が提出されたときは、会社は速やかに申出者に対し、介護短時間勤務取扱通知書(社内様式13)を交付する。その他適用のための手続等については、\ref{育介_条_介護休業の申出の手続等}から\ref{育介_条_介護休業の期間等}までの規定を準用する。

\subsubsection{介護短時間勤務の措置中の給与}
\label{育介_項_介護短時間勤務の措置中の給与}
本制度の適用を受ける間の給与については、別途定める給与規定に基づく基本給を時間換算した額を基礎とした実労働時間分の基本給と諸手当の全額を支給する。

\subsubsection{介護短時間勤務の措置中の賞与}
\label{育介_項_介護短時間勤務の措置中の賞与}
賞与については、その算定対象期間に本制度の適用を受ける期間がある場合においては、短縮した時間に対応する賞与は支給しない。

\subsubsection{介護短時間勤務の措置中の昇給・退職金}
\label{育介_項_介護短時間勤務の措置中の昇給・退職金}
定期昇給及び退職金の算定に当たっては、本制度の適用を受ける期間は通常の勤務をしているものとみなす。

\section{その他の事項}

\subsection{給与等の取扱い}
\label{育介_条_給与等の取扱い}

\subsubsection{育児・介護休業の給与}
\label{育介_項_育児・介護休業の給与}
育児・介護休業の期間については、基本給その他の月毎に支払われる給与は支給しない。

\subsubsection{育児・介護休業の賞与}
\label{育介_項_育児・介護休業の賞与}
賞与については、その算定対象期間に育児・介護休業をした期間が含まれる場合には、出勤日数により日割りで計算した額を支給する。

\subsubsection{育児・介護休業の昇給}
\label{育介_項_育児・介護休業の昇給}
定期昇給は、育児・介護休業の期間中は行わないものとし、育児・介護休業期間中に定期昇給日が到来した者については、復職後に昇給させるものとする。

\subsubsection{育児・介護休業の退職金}
\label{育介_項_育児・介護休業の退職金}
退職金の算定に当たっては、育児・介護休業をした期間を勤務したものとして勤続年数を計算するものとする。

\subsection{介護休業期間中の社会保険料の取扱い}
\label{育介_条_介護休業期間中の社会保険料の取扱い}

\subsubsection{介護休業期間中の社会保険料の取扱い}
\label{育介_項_介護休業期間中の社会保険料の取扱い}
介護休業により給与が支払われない月における社会保険料の被保険者負担分は、各月に会社が納付した額を翌月○日までに従業員に請求するものとし、従業員は会社が指定する日までに支払うものとする。

\subsection{教育訓練}
\label{育介_条_教育訓練}

\subsubsection{職場復帰プログラム}
\label{育介_項_職場復帰プログラム}
会社は、3か月以上の育児休業又は1か月以上の介護休業をする従業員で、休業期間中、職場復帰プログラムの受講を希望する者に同プログラムを実施する。

\subsubsection{職場復帰プログラムの実施}
\label{育介_項_職場復帰プログラムの実施}
会社は、別に定める職場復帰プログラム基本計画に沿って、当該従業員が休業をしている間、同プログラムを行う。

\subsubsection{職場復帰プログラムの費用負担}
\label{育介_項_職場復帰プログラムの費用負担}
同プログラムの実施に要する費用は会社が負担する。

\subsection{復職後の勤務}
\label{育介_条_復職後の勤務}

\subsubsection{復職後の勤務地・職務等}
\label{育介_項_復職後の勤務地・職務等}
育児・介護休業後の勤務は、原則として、休業直前の部署及び職務とする。

\subsubsection{復職後の勤務地・職務等の変更}
\label{育介_項_復職後の勤務地・職務等の変更}
\ref{育介_項_復職後の勤務地・職務等}にかかわらず、本人の希望がある場合及び組織の変更等やむを得ない事情がある場合には、部署及び職務の変更を行うことがある。この場合は、育児休業終了予定日の1か月前又は介護休業終了予定日の2週間前までに正式に決定し通知する。

\subsection{年次有給休暇}
\label{育介_条_年次有給休暇}

\subsubsection{年次有給休暇}
\label{育介_項_年次有給休暇}
年次有給休暇の権利発生のための出勤率の算定に当たっては、育児・介護休業をした日並びに子の看護休暇及び介護休暇を取得した日は出勤したものとみなす。

\subsection{法令との関係}
\label{育介_条_法令との関係}

\subsubsection{法令との関係}
\label{育介_項_法令との関係}
育児・介護休業、子の看護休暇、介護休暇、育児のための所定外労働の免除、育児・介護のための時間外労働及び深夜業の制限並びに所定労働時間の短縮措置等に関して、この規則に定めのないことについては、育児・介護休業法その他の法令の定めるところによる。

\fusoku{}
\subsection{附則}
本規則は、平成◯年◯月◯日から適用する。

\subsection{附則}
本規則は、平成◯年◯月◯日から適用する。


\end{document}





